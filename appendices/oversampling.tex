\documentclass{article}

\usepackage{amsmath}
\usepackage{booktabs}
\usepackage{booktabs}
\usepackage[table,xcdraw]{xcolor}

\date{\vspace{-5ex}}

\begin{document}
\title{Poststratification calculations}
\maketitle

\begin{table}[hbt!]
\centering
\footnotesize{Table 1.\textit{Population distribution of (non-)outlier primary studies.}}
\begin{tabular}{@{}ccc@{}}
\toprule
 & \textbf{no. of primary studies} &  \\ \midrule
\textbf{non-outlier} & 1370 & \cellcolor[HTML]{C0C0C0}70\% \\ \midrule
\textbf{outlier} & 581 & \cellcolor[HTML]{C0C0C0}30\% \\ \midrule
\text{Total} & \cellcolor[HTML]{C0C0C0}1951 & \cellcolor[HTML]{C0C0C0}100\% \\ \bottomrule
\end{tabular}
\end{table}

\begin{table}[hbt!]
\centering
\footnotesize{Table 2.\textit{Sample distribution of (non-)outlier primary studies.}}
\begin{tabular}{@{}cccc@{}}
\toprule
 & \textbf{no. of primary studies} &  \\ \midrule
\textbf{non-outlier} & 303 & \cellcolor[HTML]{C0C0C0}61\% \\ \midrule
\textbf{outlier} & 197 & \cellcolor[HTML]{C0C0C0}39\% \\ \midrule
\text{Total} & \cellcolor[HTML]{C0C0C0}500 & \cellcolor[HTML]{C0C0C0}100\% \\ \bottomrule
\end{tabular}
\end{table}

Because we oversampled outliers, our sample is not representative for the population. Presented in Table 1 and Table 2 are the percentages of primary studies that have effect sizes classified as (non-)outliers in the population and sample. As can be seen, 30\% of the population is classified as an outlier, compared to 39\% of our sample. This means our complete sample contains too many outlier primary study effect sizes, and too few non-outlier primary study effect sizes. First, we adjusted the sample proportions for each meta-analysis separetely, so they are in line with the population proportions of each meta-analysis. Meta-analysis 1 will be used as an example throughout the document.

\vspace{3mm}

\begin{table}[hbt!]
\centering
\footnotesize{Table 3.\textit{Population distribution of (non-)outlier primary studies in meta-analysis 1.}}
\begin{tabular}{@{}ccc@{}}
\toprule
 & \textbf{no. of primary studies in MA1} &  \\ \midrule
\textbf{non-outlier} & 37 & \cellcolor[HTML]{C0C0C0}65\% \\ \midrule
\textbf{outlier} & 20 & \cellcolor[HTML]{C0C0C0}35\% \\ \midrule
\text{Total} & \cellcolor[HTML]{C0C0C0}57 & \cellcolor[HTML]{C0C0C0}100\% \\ \bottomrule
\end{tabular}
\end{table}


\begin{table}[hbt!]
\centering
\footnotesize{Table 4.\textit{Sample distribution of (non-)outlier primary studies and frequency distribution of (non-)errors found in meta-analysis 1.}}
\newline
\begin{tabular}{@{}cccc@{}}
\toprule
 & \textbf{no error} & \textbf{error} &  \\ \midrule
\textbf{non-outlier} & 5 & 5  & \cellcolor[HTML]{C0C0C0}50\% \\ \midrule
\textbf{outlier} & 7 & 3 & \cellcolor[HTML]{C0C0C0}50\%\\ \midrule
\text{Total} & \cellcolor[HTML]{C0C0C0}12 & \cellcolor[HTML]{C0C0C0}8 & \cellcolor[HTML]{C0C0C0} 20 (100\%) \\ \bottomrule
\end{tabular}
\end{table}

As can be seen in Table 3, the proportion of outlier primary study effect sizes in meta-analysis 1 is 35\% in the population, but Table 4 shows outlier primary study effect sizes take up 50\% in the sample. Note that errors in Table 4 are classified as either a differently calculated effect, an effect that did not contain enough statistical information to reproduce, or an ambigious effect. To correct for the oversampling of outliers, we first calculated correction weights using type of effect size (outlier or non-outlier) as the auxiliary variable. As a result, the proportion of (non-)outlier primary study effect sizes in the sample will be the same as the proportion of (non-)outlier primary study effect sizes in the population. We first calculate \textit{g}$_{1}$:



\begin{equation*}
g_{h}= \frac{\frac{N_{h}}{N}}{\frac{n_{h}}{n}}  \mbox {\footnotesize, Bethlehem, Cobben \& Schouten (2011), f.8.6 } 
\end{equation*}

where we have two strata (\textit{h$_{1}$}  and \textit{h$_{2}$}, corresponding to respectively non-outlier and outlier primary study effect sizes), uppercase \textit{N}s refer to population sizes (either per stratum or in total), and lowercase \textit{n}s to sample sizes. The correction weights for meta-analysis 1 are:

\begin{equation*}
g_{1}= \frac{\frac{N_{1}}{N}}{\frac{n_{1}}{n}} =   \frac{\frac{37}{57}}{\frac{10}{20}}= 1.2982456
\end{equation*}

\begin{equation*}
g_{2}= \frac{\frac{N_{2}}{N}}{\frac{n_{2}}{n}} =  \frac{\frac{20}{57}}{\frac{10}{20}} = 0.7017544
\end{equation*}

Since we do not have any information on the number of errors in the population, we assume the same weights for non-errors and errors within each of the strata. As such, we multiply the first row of Table 4 with \textit{g}$_{1}$, and the second row of Table 4 with \textit{g}$_{2}$:

\begin{table}[hbt!]
\centering
\footnotesize{Table 5.\textit{Sample distribution of (non-)outlier primary studies and frequency distribution of (non-)errors found in meta-analysis 1, corrected.}}
\newline
\begin{tabular}{@{}cccc@{}}
\toprule
 & \textbf{no error} & \textbf{error} &  \\ \midrule
\textbf{non-outlier} & 6.49 & 6.49  & \cellcolor[HTML]{C0C0C0}65\% \\ \midrule
\textbf{outlier} & 4.91 & 2.10 & \cellcolor[HTML]{C0C0C0}35\%\\ \midrule
\text{Total} & \cellcolor[HTML]{C0C0C0}11.40 & \cellcolor[HTML]{C0C0C0}8.60 & \cellcolor[HTML]{C0C0C0} 20 (100\%) \\ \bottomrule
\end{tabular}
\end{table}

As Table 5 shows, the sample distribution of meta-analysis 1 is now proportional to the population distribution of meta-analysis 1. However, these proportional weights do not expand results to population size, only the proportions are restored (Maletta, 2007). To generalize to the population, inclusion weights \textit{d$_{h}$} per stratum and per meta-analysis are calculated. The inclusion weight is the inverse of the inclusion probability (\textit{n} $/$ \textit{N}):

\begin{equation*}
d_{1}= \frac{N_{1}}{n_{1}} = \frac{37}{10} = 3.7
\end{equation*}

\begin{equation*}
d_{2}= \frac{N_{2}}{n_{2}} = \frac{20}{10} = 2
\end{equation*}

This means two outlier primary study effect sizes in the population are represented by one outlier primary study effect size in the sample. We multiply the first row of Table 5 with \textit{d$_{1}$}, and the second row with \textit{d$_{2}$}:

\begin{table}[hbt!]
\centering
\footnotesize{Table 6.\textit{Sample distribution of (non-)outlier primary studies and frequency distribution of (non-)errors found in meta-analysis 1, corrected.}}
\newline
\begin{tabular}{@{}cccc@{}}
\toprule
 & \textbf{no error} & \textbf{error} &  \\ \midrule
\textbf{non-outlier} & 24.02 & 24.02  & \cellcolor[HTML]{C0C0C0}77\% \\ \midrule
\textbf{outlier} & 9.82 & 4.21 & \cellcolor[HTML]{C0C0C0}23\%\\ \midrule
\text{Total} & \cellcolor[HTML]{C0C0C0}33.84 & \cellcolor[HTML]{C0C0C0}28.23 & \cellcolor[HTML]{C0C0C0} 62.07 (100\%) \\ \bottomrule
\end{tabular}
\end{table}

The total number of studies in meta-analysis 1, presented in Table 6 (62), approximately corresponds to the population total number of studies of meta-analysis 1 (57). Since the inclusion weight of a non-outlier primary study effect size is larger than that of an outlier primary study effect size, the non-outliers receive more weight.





The adjustment weights w$_{h}$ are obtained by multiplying the correction weights g$_{h}$ by the inclusion weight d$_{h}$ (Bethlehem, Cobben \& Schouten (2011), p.215). Since we do not have any data on errors in the population of primary studies, we use the same weight for both non-errors and errors per stratum, and multiply each of the four cells in Table 3 with these weights (i.e., the non-outlier estimates with \textit{w}$_{1}$, and the outlier estimates with \textit{w}$_{2}$, resulting in the estimates presented in Table 4.

\begin{equation*}
w_{1}=g_{1} \times d_{1}= 5.239244
\end{equation*}

\begin{equation*}
w_{2}=g_{2} \times d_{2}= 2.229115
\end{equation*}

\begin{table}[hbt!]
\centering
\ \ \ \ \ \ \ \ \ \footnotesize{Table 4.\textit{Frequencies of (non-)errors found in primary studies\ \ \ \ \ \ \ \ \ \ \ \ \ \ \ \ \ \ \ \ \ \ \ \ \  classified as (non-)outlier, weight corrected.}}
\newline
\begin{tabular}{@{}cccc@{}}
\toprule
 & \textbf{no error} & \textbf{error} &  \\ \midrule
\textbf{non-outlier} & 817.3 & 628.7 & \cellcolor[HTML]{C0C0C0}1446.0 \\ \midrule
\textbf{outlier} & 327.7 & 171.6 & \cellcolor[HTML]{C0C0C0}449.3\\ \midrule
\text{Total} & \cellcolor[HTML]{C0C0C0}1145.0 & \cellcolor[HTML]{C0C0C0}800.4 & \cellcolor[HTML]{C0C0C0} 1945.4 \\ \bottomrule
\end{tabular}
\end{table}

We used the estimates from Table 4 to calculate the conditional probabilities of finding an error (i.e., either a different, incomplete, or ambiguous effect (size)) in a primary study, given that you either have a primary study effect size that is classified as a non-outlier or outlier.

\begin{equation*}
P_{err} = \frac{800.4}{1945.4} = 0.4114169
\end{equation*}

\begin{equation*}
P_{(err|non-outlier)} =  \frac{P_{(non-outlier \ and \ err)}}{P_{(non-outlier)}}  =  \frac{\frac{628.7}{1945.4}}{\frac{1446.0}{1945.4}} = 0.4347826
\end{equation*}

\begin{equation*}
P_{(err|outlier)} =  \frac{P_{(outlier \ and \ err)}}{P_{(outlier)}}  =\frac{\frac{171.6}{1945.4}}{\frac{499.3}{1945.4}}= 0.34375
\end{equation*}

\begin{equation*}
P_{(non-outlier|err)} =  \frac{P_{(err \ and \ non-outlier)}}{P_{(err)}}  =\frac{\frac{628.7}{1945.4}}{\frac{800.4}{1945.4}}=  0.7855418
\end{equation*}

\begin{equation*}
P_{(outlier|err)} =  \frac{P_{(err \ and \ outlier)}}{P_{(err)}}  =\frac{\frac{171.6}{1945.4}}{\frac{800.4}{1945.4}}=  0.2144582
\end{equation*}

Before correction:\newline
0.394 \newline
0.4347826 (same) \newline
0.34375 (same) \newline
0.6091371 \newline
0.3908629 \newline
Bespreken d gewicht goed?
Als ik 1 d gewicht gebruik (N$/$n), kom ik hierop uit:\newline
    no error    error\newline
reg 705.3465 542.5743\newline
out 433.5381 227.0914\newline
 p.x\newline
[1] 0.4032724\newline
 p.x.a \newline
[1] 0.4347826\newline
 p.x.b \newline
[1] 0.34375\newline
 p.a.x \newline
[1] 0.704948\newline
 p.b.x \newline
[1] 0.295052\newline
totaalmat\newline
 1908.55\newline
p.243

------ einde tekst hier -------

We gaan er nu wel van uit dat de kans dat je een error vindt gelijk is over MAs, terwijl er ook hele goede en hele slechte MAs tussen zitten, en de kans tussen MAs met correlatie en MAs met smd ook nogal verschilt.

Since outlier and non-outlier are not independent with regards to x, the conditional probabilities are dependent on each other.


With these estimates, it is possible to calculate the probability of finding an error in a primary study effect size, given that you are dealing with either a non-outlier or outlier primary study effect size.


\begin{equation*}
P_{non-outlier} = \frac{(817.32 + 628.71)}{1945.35} = 0.7433259
\end{equation*}
\begin{equation*}
P_{outlier} = \frac{(327.68 + 171.64)}{1945.35} = 0.2566741
\end{equation*}



\vspace{3mm}

\end{document}